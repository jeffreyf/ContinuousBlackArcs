\documentclass[12pt,t,xcolor=dvipsnames]{beamer}
\usepackage{amsmath,amssymb}
\usepackage{multirow}
\usepackage{subfigure}
\usepackage{pgfplots}


% Beamer Commands
\setbeamertemplate{navigation symbols}{}
\setbeamertemplate{footline}
{%
\hspace*{0.7\linewidth}\insertshorttitle - p.\insertframenumber
}
\setbeamertemplate{footnote}
{%
\insertfootnotetext
}
\setbeamercolor{footnote mark}{fg=white}
\setbeamertemplate{frametitle}[default][center]
\setbeamertemplate{itemize item}[circle]
\setbeamertemplate{itemize subitem}[triangle]
\setbeamercolor{itemize subitem}{fg=Plum}
\setbeamerfont{itemize subitem}{size=\normalsize}
\setbeamercolor{alerted text}{fg=Magenta}
\setbeamerfont{institute}{size=\normalsize}
\setbeamerfont{list label}{series=\bfseries}
\usefonttheme[onlylarge]{structurebold} 

\newcommand{\alerta}[1]{{\usebeamercolor[fg]{frametitle} #1}}

% Title Page Stuff
\title{Progress Report on Black Arcs Problem}
\date{July 4, 2018}


\begin{document}

% Title Page
% - Begin Slide -----
\maketitle

% - Begin Slide -----
\begin{frame}
  \frametitle{Problem Summary}
  \onslide*<1>{
\vspace*{-12pt}
  \begin{center}
    Natural graphs of city layouts don't ``look nice''\\[12pt]

    \includegraphics[width=\textwidth]{original_sackville}
  \end{center}}%
  \onslide*<2>{
    What does it mean to ``look nice''?
    \begin{itemize}
    \item Heuristic idea: sharp angles
    \item Ideal to have small set of allowable angles
    \item We take these to be multiples of $\pi/8$
    \end{itemize}

    Problem: how to translate graph nodes/edges to make angles more
    like multiples of $\pi/8$?
    }

\end{frame}

% - Begin Slide -----
\begin{frame}
  \frametitle{The Data}

  Get map data in list format:
  \begin{itemize}
  \item List of $n_v$ vertices as $(x_i,y_i)$ coordinates
  \item List of $n_e$ edges as vertex pairs
  \end{itemize}

  \begin{center}
    Look to translate nodes $(x_i,y_i) \rightarrow (\tilde{x}_i,\tilde{y}_i)$ so
    to ``improve'' angles formed by edges
  \end{center}
\pause
  Three principles:
  \begin{enumerate}
  \item Don't let nodes move a lot
  \item Don't let angles change a lot
  \item Make angles equal to $k\pi/8$ for some $k\in\mathbb{Z}$
  \end{enumerate}


\end{frame}


% - Begin Slide -----
\begin{frame}
  \frametitle{Continuous Optimization Approach}

  \begin{center}
    Central idea: relax constraint
  \end{center}

  Compose {\it fitness function} from three terms
  \begin{enumerate}
  \item Penalize node movement: $\displaystyle\sum_{i=1}^{n_v}
    \left((\tilde{x}_i-x_i)^2+(\tilde{y}_i - y_i)^2\right)$
  \item Penalize angle change: $\displaystyle\sum_{j=1}^{n_e} \left(\tilde{\theta}_j -
    \theta_j\right)^2$
    \begin{itemize}
    \item Given edge $e_j = \left\{(x_k,y_k),(x_\ell,y_\ell)\right\}$,
      define $\theta_j = \text{arctan2}(x_\ell-x_k,y_\ell-y_k)$
      \item ``Four-quadrant inverse tangent''
    \end{itemize}
    \item Penalize difference from $k\pi/8$: $\displaystyle\sum_{j=1}^{n_e}\sin^2\left(8\tilde{\theta}_j\right)$
  \end{enumerate}
  
\end{frame}

% - Begin Slide -----
\begin{frame}
\frametitle{Fitness Function}

Take weighted sum of these:
\begin{align*}
f(\tilde{x},\tilde{y}) = \alpha&\sum_{i=1}^{n_v}
\left((\tilde{x}_i-x_i)^2+(\tilde{y}_i - y_i)^2\right)\\
& + \beta\sum_{j=1}^{n_e} \left(\tilde{\theta}_j - \theta_j\right)^2
+\gamma\sum_{j=1}^{n_e}\sin^2\left(8\tilde{\theta}_j\right)
\end{align*}

Problem: How to choose $\alpha$, $\beta$, $\gamma$?
\begin{itemize}
\item First pass: experiment!
  \begin{itemize}
  \item Choosing equal weights leads to difficult optimization
  \item Moving nodes a lot (in coordinate space) is okay
  \item Roughly $\alpha = 10^{-5}$, $\beta = \gamma = 1$
  \end{itemize}
\end{itemize}

\end{frame}

% - Begin Slide -----
\begin{frame}
  \frametitle{Black-Box Optimization}

  Many black-box optimization tools available
  \begin{itemize}
  \item Derivative-free optimization
    \begin{itemize}
    \item Nelder-Mead, Powell, \ldots
    \item Try to identify search direction, go downhill
    \end{itemize}
  \item Basin-hopping
    \begin{itemize}
    \item Combine these with local searches above
    \end{itemize}
  \item Differential Evolution
    \begin{itemize}
    \item Metaheuristic method
    \item Create population of candidate solutions, recombine to
      improve optimality
    \end{itemize}
  \item Many others
    \begin{itemize}
      \item Chose to use just those available in scipy
    \end{itemize}
  \end{itemize}

\end{frame}

% - Begin Slide -----
\begin{frame}
  \frametitle{First results}
\begin{center}
  \includegraphics[width=\linewidth]{map1}
\end{center}
\end{frame}

% - Begin Slide -----
\begin{frame}
  \frametitle{Second results}
\begin{center}
  \includegraphics[width=\linewidth]{map2}
\end{center}
  

\end{frame}

% - Begin Slide -----
\begin{frame}
  \frametitle{Measuring Quality}

  Visual quality seems good, but what about some numbers?
  \onslide*<1>{
  \begin{itemize}
  \item For first example
    \begin{itemize}
    \item Reduce fitness function from 6.35 to 0.67
    \item Reduce scaled deviation from angles of $k\pi/8$ from 0.45
      to 0.10
    \end{itemize}
  \end{itemize}}%
  \onslide*<2>{
    \begin{itemize}
    \item For second example
      \begin{itemize}
      \item Reduce fitness function from 21.29 to 1.61
      \item Reduce scaled deviation from angles of $k\pi/8$ from 0.48
        to 0.15
      \end{itemize}
    \end{itemize}

  \begin{center}
    \includegraphics[width=0.48\linewidth]{../pngs/original_deviation_histogram}
    \hspace*{0.01\linewidth}
    \includegraphics[width=0.48\linewidth]{../pngs/optimized_deviation_histogram}
  \end{center}}%
    
        
\end{frame}

% - Begin Slide -----
\begin{frame}
  \frametitle{Some choices}

  \begin{center}
    \includegraphics[width=0.48\linewidth]{../pngs/closer_terms}
    \hspace*{0.01\linewidth}
    \includegraphics[width=0.48\linewidth]{../pngs/stuckout_term2}
  \end{center}
    
  Two results from slightly different optimizer options
  \begin{itemize}
    \item Graph at right has much lower deviation in angles from
      $k\pi/8$
    \item Achieving lower deviation comes at trade-off with node displacement
  \end{itemize}
\end{frame}

% - Begin Slide -----
\begin{frame}
  \frametitle{Next steps}

  Working with Sackville map adds some complexity
  \begin{itemize}
  \item Direct application of optimization approach only reduces maximum
    deviation from 0.50 to 0.42
  \end{itemize}

  Idea: rotate first, then optimize relative to rotated map
  \begin{itemize}
  \item Eyeball $23^\circ$ rotation, from histogram of initial angles
    \begin{itemize}
    \item Can we automate this reliably?
    \end{itemize}
  \item Working on optimization after rotation
  \end{itemize}

\end{frame}

% - Begin Slide -----
\begin{frame}
  \frametitle{Hot off the press!}

  \onslide*<1>{
    \begin{center}
      \includegraphics[width=\linewidth]{../pngs/map3_optimized_after_rotation}
    \end{center}
  }%
  \onslide*<2>{
  \begin{center}
    \includegraphics[width=0.48\linewidth]{../pngs/map3_orig_deviations}
    \hspace*{0.01\linewidth}
    \includegraphics[width=0.48\linewidth]{../pngs/map3_optimized_deviations}
  \end{center}}%
  

\end{frame}

\begin{frame}
  \frametitle{(Integer) Linear Models}
  \begin{minipage}[l]{0.45\textwidth}
    \includegraphics[width=\textwidth]{snapping.pdf}
    \end{minipage}
    \begin{minipage}[r]{0.48\textwidth}
    \begin{align*}
      \min \sum_{(i,j) \in E} |\epsilon_{i,j,1}| +& |\epsilon_{i,j,2}| \\
      \tilde{v}_i + \ell_{i,j}\sum_{d\in 1\dots 4}c_{i,j,d}w_d &= v_j+\epsilon_{i,j}\\
      c_{i,j,d} &\in \{0, 1\}\\
      \sum_{d} c_{i,j,d} &= 1\\
      \ell_{i,j} &= \| v_i - v_j\|
    \end{align*}
  \end{minipage}

\end{frame}

\begin{frame}
  \frametitle{Linear Models}
  \begin{minipage}[l]{0.45\textwidth}
    \includegraphics[width=\columnwidth]{snapping2.pdf}
    \end{minipage}
    \begin{minipage}[r]{0.48\textwidth}
    \begin{align*}
       \max \sum_{(i,j) \in E} \langle z_{i,j} &, w^*_{i,j} \rangle\\
      z_{i,j} & = (\tilde{v}_{i} - \tilde{v}_{j})/l_{i,j}\\
       w^*_{i,j}&= \text{closest $w_d$}\\
       \ell_{i,j} &= \| v_i - v_j\|
    \end{align*}
    In both cases we need penalties/bounds to prevent vertices from
    moving too far.
  \end{minipage}
\end{frame}

\end{document}

% - Begin Slide -----
\begin{frame}
\frametitle{}

\end{frame}

% - Begin Slide -----
\begin{frame}
\frametitle{}

\end{frame}
